\documentclass{argres}
% \documentclass{argres_continfo}

% \quote{Chemical Engineering student with research experience and a strong computer programming background}

\begin{document}

\makecvtitle

\vspace*{-10mm}

\section{Education}

\cventry{2017--present}{Ph.D. Chemical Engineering}{University of Florida}{Gainesville, FL}{}{}
\cventry{}{M.S. Chemical Engineering}{University of Florida}{Gainesville, FL}{}{}
\cventry{}{B.S. Chemistry}{University of Miami}{Coral Gables, FL}{}{}

\section{Skills}
\cvitem{Programming}{Python, Git, \LaTeX, MATLAB, JavaScript, Visual Basic}
\cvitem{Software}{ Minitab, ImageJ, Office Suites, VESTA, Diamond, LIMS}
% \cvitemwithcomment{Languages}{Spanish and English}{native/bilingual proficiency}

\section{Experience}

\cventry{08/2017--present}{Graduate Assistant}{\textsc{University of Florida}}{Gainesville}{}{
\begin{itemize}
% \item Design of a novel reactor system to be used for a crystal synthesis.
% 	\begin{itemize}
% 	\item Drafted a schematic of the reactor setup with all the components needed for it to work.
% 	\item Reviewed physical constraints of equipment to be bought from manuals and product catalogs.
% 	\item Communicated with vendors through phone calls to obtain price quotes and information from desired supplies.
% 	\end{itemize}
\item Research Assistant - Topic: Transport measurements of gas diffusion through metal organic framework (MOF) channels and synthesis thereof.
	% \begin{itemize}
	% \item Designing and executing experiments based on a working theoretical understanding of crystal synthesis.
	% \item Analyzing crystal structures through the use of X-Ray Diffraction (XRD) and SEM.
	% \item Collaborating with fellow researchers to characterize crystals through additional methods pertinent to our investigation's aims.
	% \end{itemize}
\item Teaching assistant for the Computer Model Formulation(COT3502) and Chemical Kinetics and Reactor Design (ECH4504)
chemical engineering courses for the Fall 2017--Spring 2018 semester.
% \item TA for Computer Model Formulation(COT3502)s: class of 75 students (Spring 2018)
% 	\begin{itemize}
% 	\item Holding office hours every week, instructing students on the integration of the Python programming language for numerical analysis.
% 	\item Individually grading all evaluations (quizzes and exams) for the class given professor's grading rubric.
% 	\end{itemize}
% \item TA for Chemical Kinetics and Reactor Design (ECH4504): class of 114 students (Fall 2017)
% 	\begin{itemize}
	% \item Held office hours every week and exam reviews for first and second exams \item Individually graded all exams for the class and answered questions regarding grading
% 	\end{itemize}
\end{itemize}}


\cventry{07/2015--07/2017}{Associate Engineer}{\textsc{Xerox}}{Webster}{NY}{
\begin{itemize}
\item Optimized toners to meet standards demanded by client(s) by conducting/analyzing several factorial experiments (DOEs)
\item Contributed to Xerox's intellectual property portfolio by passing invention submissions focused on toner quality improvement.
% \item Designed chemical toners for Xerox, leaving the company after 2017s 2nd quarter (operating cash flow: \$343 million, \
% up \$84 million from the same period in 2016).
\end{itemize}}


\cventry{12/2013--06/2015}{Research Assistant}{\textsc{University of Florida}}{Gainesville}{}{
\begin{itemize}
\item Adapted a process which was used to support the submission of a \$45,000 commercialization proposal in June 2014.
\item Materialized primary goals of research project, making helpful contributions to the passing of a larger
\href{http://grantome.com/grant/NIH/R21-NS093239-02}{\textbf{NIH R01 funded}} (about \$180,000 for the year 2015) project.\
\item Co-invented a technology highly applicable to the \$1.68 billion dollar market of nerve repair and regeneration.
\end{itemize}}


\section{Data Science / Programming Exp. and Projects}

\cvitem{2018--present Ph.D. Research (University of Florida)}{
\newline
{o Development of a kinetic Monte Carlo (kMC) algorithm for a MOF crystallization process (Python) and an analytical expression thereof.
\newline
o Development of chemical equilibrium reaction networks to model concentration of chemical species with respect to system changes based on a system of non-linear equations solver (Python)} }
\cvitem{Spring 2018 COT3502 TA (University of Florida)}{
\newline
{o Assisted students with programming development (Python) and theoretical understanding; scripts returning calculations and data structures
through the implementation of logical expressions, as well as solving equations through
the use of numerical analysis theory (Gaussian elimination, Newton-Raphson, Runge-Kutta, etc.)  } }
\cvitem{2015-2017 Engineer (Xerox)}{
\newline
{o Provided estimates of spread (standard deviation) through factorial and simple Monte Carlo (sMC) methods for a system level design which was implemented at the production scale.
\newline
o Created a Monte Carlo simulation algorithm through Python which predicted the results of a \textbf{non-disclosed} characterization method.
Algorithm was also translated to Excel in a low-level manner by tabulating functions into cells. If implemented it would be saving the company some valuable employee time.
} }

\cvitem{2014-2015 M.S. Research (University of Florida)}{
\newline
\href{https://doi.org/10.1016/j.colsurfa.2017.05.058}
{o Proposed a model for crosslinked microsphere size from power-law fits through Minitab and published an \textbf{article} thereof based on fundamental theory}
\newline
\href{https://github.com/andrewrgarcia/scipycon_15}
{o Prepared all publication plots using Python and submitted an \textbf{entry} to the 2015 Scipy John Hunter plotting contest
}  }
\cvitem{Fall 2014 M.S. Student (University of Florida)}{
\newline
{o Learned Python and worked on \textbf{3} projects which integrated Python to solve problems on statistical mechanics:
\newline
(1) Calculations of a polymer's hydrodynamic radius from random walk theory, making use of probability density functions.
\newline
(2) Simulation of a 2-dimensional Ising Model.
\newline
(3) Calculations of Internal Energy and Pressure from Monte Carlo simulations of a 2-D Lennard-Jones Model.  } }

\section{Publications}
% \cventry{}{}{}{}{}{
% \begin{itemize}
\cvitem{1}{ \href{https://doi.org/10.1016/j.colsurfa.2017.05.058}
{AR Garcia, C Lacko, C Snyder, AC Bohorquez, CE Schmidt, C Rinaldi. (2017) "Processing-size correlations in the preparation of
magnetic alginate microspheres through emulsification and ionic crosslinking" \textit{Colloids Surf., A}. 529:119-127}}
\cvitem{2}{ \href{http://uf.catalog.fcla.edu/permalink.jsp?20UF033653171}
{AR Garcia (2015) "Synthesis of dissolvable magnetic microspheres for tissue scaffold applications (MS Thesis)"
\textit{University of Florida}}}
\cvitem{3}{ \href{http://www.sciencedirect.com/science/article/pii/S0927776513000180}
{AR Garcia, I Rahn, S Johnson, R Patel, J Guo, J Orbulescu, M Micic, JD Whyte, P Blackwelder, RM Leblanc.(2013)
"Human insulin fibril-assisted synthesis of fluorescent gold nanoclusters in alkaline media under physiological temperature"
\textit{Colloids Surf., B}. 105:167-172}}
\cvitem{4}{ \href{https://doi.org/10.1021/la204201w}
{W Liu, S Johnson, M Micic, J Orbulescu, JD Whyte, AR Garcia, RM Leblanc.(2012) "Study of the aggregation of human
insulin langmuir monolayer"
\textit{Langmuir}. 28(7):3369–3377}}
% \end{itemize}
% }

\section{Patents and Inventions}

% \begin{itemize}
\cvitem{2018}{ \textsc{Patent:} \href{https://patents.google.com/patent/US20180133372A1}{C Rinaldi, CE Schmidt, C Lacko, Z Khaing, AR Garcia
 "Magnetically templated tissue engineering scaffolds and methods of making and using the magnetically templated tissue
 engineering scaffolds" \textbf{US Patent} \textit{\textbf{US20180133372A1}}, PCT filed May 11, 2016} }

\cvitem{2016-2017}{\textsc{Xerox Trade Secrets:} (6 total) Primary author of \textbf{5} }

% \cvitem{2016}{ \textsc{Patent:} \href{http://google.com/patents/WO2016183162}{C Rinaldi, CE Schmidt, C Lacko, Z Khaing, AR Garcia
%  "Magnetically templated tissue engineering scaffolds and methods of making and using the magnetically templated tissue
%  engineering scaffolds" \textbf{PCT Patent} \textit{\textbf{WO2016183162 A1}}, issued November 17, 2016} }

% \href{http://technologylicensing.research.ufl.edu/technologies/15266_magnetically-aligned-nerve-guide-scaffold-for-safe-comprehensive-peripheral-nerve-regeneration}
% {"Magnetically Templated Tissue Engineering Scaffolds and Methods of Making and Using the Magnetically Templated Tissue Engineering Scaffold"} \\
% % \textbf{Magnetically Templated Tissue Engineering Scaffolds and
% % Methods of Making and Using the Magnetically Templated Tissue Engineering Scaffolds} \\
% \textit{Technology \#15266} \hfill  \textit{Filing Date: May 12, 2015}
% % \textit{Serial No.: 62/160,202} \hfill  \textit{Filing Date: May 12, 2015}


\section{Certificates}
\subsection{Technical}
\cvitem{09/2016}{ \textbf{Design for Six Sigma IDOV Green Belt}, Xerox}
\cvitem{2015--2018}{\textbf{Lean Six Sigma DMAIC Green Belt}, 2221-4545, IIE}
\cvitem{07/2013}{ \textbf{Process Engineering Certificate}, University of Florida}

\subsection{First Aid}
\cvitem{2015--2017} {\textbf{Healthcare Provider}, NY15657, American Heart Association}
\cvitem{2015--2017}{\textbf{Heartsaver\textregistered First Aid}, NY15657, American Heart Association}
% \item \textbf{Heartsaver\textregistered CPR AED}, TN20087, American Heart Association \hfill            04/2015--04/2017
% \item \textbf{Blood-Borne Pathogens and Oxygen Therapy}, Action CPR LLC \hfill            				Issued 12/2015
\end{document}
